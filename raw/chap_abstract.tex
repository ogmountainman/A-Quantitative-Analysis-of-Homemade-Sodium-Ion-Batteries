\section{Abstract}
Electricity is based on the movement of charged matter. Through the work done upon a moving charge in an electric field, a potential emerges. Through the formation of a circuit, the potential causes a current to flow and consequently energy to be transmitted\cite{Kammer2019}.\\
Electric energy powers most of modern infrastructure and has become an essential pillar of almost every facet of modern life. With the only limitation being the need of electric outlets. However, with the development of the battery, mobile electronic devices have taken over the modern lifestyle and made the battery one of the most important components in the general field of electronics. Nonetheless, the current rechargeable, consumer battery market is dominated by lithium-ion technology, which comes with a variety of disadvantages, such as the environmental impacts of lithium, the danger of thermal runaway and the depletion of resources\cite{Peters2017, Peters2016, Wang2012}. This experiment investigates the construction process and the viability of a sequence of homemade sodium-ion batteries. The batteries are constructed using a graphite anode and a titanium-dioxide cathode separated by a layer of tape, with a 1 mol/L sodium-perchlorate aqueous solution acting as the electrolyte. In separate trials with varying cathode thickness, anode substance and battery form-factor, the battery voltage, discharge current, as well as multiple discharge curves are determined. The results are tabulated and compared with a state-of-the-art lithium battery, while taking into consideration the limitations of the experiment. The average capacities achieved are \SI{5.52e-3}{\mA\hour} and \SI{3.18e-3}{\mA\hour} with maximum power outputs of \SI{1.01e-3}{\W} and \SI{3.30e-4}{\W}. While this does not currently outperform any commonly available battery, the value of this experiment lies in the proof of concept: Detailing the construction of an easily built, cheap, and safe alternative to the common lithium-ion battery. With only minimal investment, the technology presented could prove a sustainable future for batteries.