\section{Safety Guidelines}
When working with electric circuits, the danger of getting electrocuted is always present, particularly when working with open circuits and self-made batteries, since the chance of touching a live component is increased. However, since the power of the batteries constructed is insignificant, this danger is limited when working with the battery alone. During the charging process on the other hand, special attention must be paid, since the battery is connected to a constant \SI{4.6}{\V} which can be lethal at high enough amperage. This is again limited by the modern security equipment present in every laboratory, with fuses protecting the connections from drawing too much current upon a touch.\\
Furthermore, some of the substances used require special attention. The \SI{1}{\mol\per\L} nitric acid utilized to create titanium-dioxide suspensions must be exclusively handled with gloves and disposed of correctly. Moreover, the titanium-dioxide powder in its raw form is labelled as a potential occupational carcinogen when inhaled. When handling the powder in its pure form, a fume hood prevents contact with the respiratory system \cite{Prevention2019}. However, since the powder was mostly handled in a suspended form, these dangers only apply during the suspension creation process.\\
Finally, caution is required when handling sodium-perchlorate since it is a reactive chemical and an explosion hazard when dried-out. Thus, the anhydrous sodium-perchlorate, as well as the electrolyte solution used were always kept in firmly closed, air-tight containers to prevent drying-out.
