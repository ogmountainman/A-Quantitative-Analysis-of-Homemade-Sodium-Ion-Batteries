\section{Discussion}
The results obtained align with the expectations. A battery with a minimal capacity and comparatively small power output was constructed. Considering both the precision of the methods applied, as well as the battery size, these values are not surprising. 
However, various limitations must be taken into consideration. Firstly, the materials found in a school laboratory are commonly lower-grade substances of unknown origin and age. In this case, the titanium-dioxide powder and the sodium-perchlorate utilized were of unknown quality and most likely impacted the results. Furthermore, the measurement tools used, particularly the multimeter, was only able to measure the voltage to a precision of two figures below zero. Thus, when working with such small values, imprecision is inevitable. Additionally, important results of modern battery construction processes, such as the titanium-dioxide lattice shape and size are not obtainable in a school laboratory. Thus, a certain degree of error is unavoidable.\\
Moreover, the initial idea of constructing a conductive cathode misled the initial experimentation phase, since it was later discovered, that a conductive titanium-dioxide layer is not strictly required for a functioning battery. However, this might be attributable to a mismeasurement of the multimeter or due to other factors, such as the lack of an electrolyte present during measurement. Additionally, the use of an aqueous electrolyte severely limited the battery performance. The recommended mixture of organic, polar substances was not available in a school setting and de-ionized water was the simplest substitute. However, the electrolysis of water taking place at any voltage above \SI{1.5}{\V} caused the destruction of an FTO slide and the constant degradation of the electrodes throughout the charging process. This can be considered the most severe limitation.\\
Finally, the discharge curve measurement was done entirely manually introducing further sources of error. Additionally, a sub-optimal load resistance was chosen. A higher resistor of \SI{1e6}{\ohm} to \SI{1e8}{\ohm} would have been better suited to a battery of this capacity, since it would have prevented the drastic initial voltage drop seen in all curves of Fig. \ref{fig:dischargecurves}.

   